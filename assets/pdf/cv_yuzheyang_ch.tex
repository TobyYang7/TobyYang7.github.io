\documentclass[a4paper,10pt]{article}
\usepackage[left=0.3in,right=0.3in,top=0.3in,bottom=0.5in]{geometry}
\usepackage{enumitem}
\usepackage[hidelinks]{hyperref}
\usepackage{xcolor}
% \usepackage{xeCJK}
\usepackage{ctex}
\usepackage[english]{babel}
\usepackage{fancyhdr}
\usepackage{datetime}
\usepackage{bibentry}
\usepackage{natbib}
\usepackage{titlesec}

\newdateformat{monthyearformat}{\shortmonthname[\THEMONTH], \THEYEAR}
\pagestyle{fancy}
\renewcommand{\headrulewidth}{0pt}
\rfoot{\scriptsize \textit{Last updated: \monthyearformat\today}}
\lfoot{\scriptsize \textit{yuzheyang@link.cuhk.edu.cn}}
\fancyfoot[C]{\scriptsize \thepage}

\pagestyle{empty}

\usepackage{setspace}
\setstretch{1.1}

\titleformat{\section}{\Large\bfseries\color{darkblue}}{}{0em}{}[\titlerule]
\titlespacing*{\section}{0pt}{0.5em}{0.4em}

\setlength{\parindent}{0pt}

% Define custom colors
\definecolor{darkblue}{RGB}{26,13,171}
\definecolor{gray}{RGB}{128,128,128}

% Customize section command
\usepackage{titlesec}
\titleformat{\section}{\bfseries\color{darkblue}}{}{0em}{}[\titlerule]

% Customize itemize environment
\setlist[itemize]{left=1.5em,label=--,itemsep=0.5em}

% Customize hyperlinks
\hypersetup{
    % colorlinks=true,
    linkcolor=darkblue,
    filecolor=magenta,
    urlcolor=darkblue,
}

% Define custom commands for personal information
% \newcommand{\name}[1]{\noindent\textbf{\LARGE #1}\vspace{0.5em}\hrule\vspace{1em}}
\newcommand{\contact}[2]{\noindent\textcolor{gray}{#1:} #2}

\begin{document}

\newcommand{\name}[1]{\noindent\textbf{\LARGE #1}\vspace{0.5em}}

\begin{center}
    
\name{Yuzhe Yang | 阳雨哲}

\contact{Email}{yuzheyang@link.cuhk.edu.cn} | \contact{Phone}{+86 18310762536} | \contact{Homepage}{\href{https://tobyyang7.github.io/}{tobyyang7.github.io}}

\vspace*{.2em}
\textbf{A 3rd-year CSE student with a keen interest in deep learning. Currently exploring GCN, LLM, and NLP.\\}
% \contact{Avaliable Time}{Jan 2024 - Mar 2024}

\end{center}

\section*{教育经历}
\textbf{数据科学学院} | \textbf{香港中文大学(深圳)} \hfill \textit{2021年9月 - 2025年5月}\\
\textbf{工学士:} 计算机科学与工程\\
\textbf{主要课程:}
数据结构 | 操作系统 | 计算机体系结构 | 机器学习 | 最优化 | 自然语言处理

\section*{技能}
\textbf{编程语言:} Python | PyTorch | C++ | RISC-V | HTML | JavaScript | React\\
\textbf{技术:} Git | VS Code | MATLAB | \LaTeX | Linux | CLI\\
\textbf{语言:} 英语 | 普通话

\section*{发表论文}
\href{https://www.sciencedirect.com/science/article/pii/S1566253524001040}{\textbf{FAST-CA: Fusion-based Adaptive Spatio-Temporal Learning with Coupled Attention for Airport Network Delay Propagation Prediction, Information Fusion, 2024, 107:102326 [online]}} \hfill \textit{2023年8月 - 2023年11月}\\
\textit{本科研究助理,指导教师 \textbf{\href{https://sds.cuhk.edu.cn/en/teacher/268}{Prof.\@ Jianfeng Mao}}} \hfill SDS, CUHK(SZ)\\
- 改进了机场网络延迟预测的深度学习模型\\
- 实现了基线模型并测量所提出模型的性能\\
- 时空数据分析和可视化分析\\
- 机场延迟预测领域的SOTA(最优)模型\\
- 深度学习,图神经网络,PyTorch,PyTorch Geometric

\section*{研究经历}
\textbf{Research in Continuous Spatio-Temporal Graph} \hfill \textit{2024年1月 - 现在}\\
\textit{本科研究助理} \hfill SDS, CUHK(SZ)\\
- 实施了一个用于交通流预测的条件时空图模型\\
- 提出了一种使用常微分方程构建连续图的新方法\\
- 时间卷积图神经网络\\
\textbf{Deep Learning Approach for Early Predicting and Controlling Network Flow in SDN} \hfill \textit{2024年1月 - 2024年5月}\\
\textit{研究实习生} \hfill ICNLAB, PKU(SZ)\\
- 使用改进的Informer架构为软件定义网络开发了一种新的网络流预测方法\\
- 设计并实施了一种基于预测的主动拥堵管理策略\\
- 在模拟环境中进行实际实验以验证所提方法的有效性\\
- 深度学习,时间序列分析,PyTorch

\section*{项目经验}
\textbf{多条件AHP光污染评估模型 | MCM} \hfill \textit{2023年2月}\\
- GIS数据分析,数学建模\\
- 通过人口数据、区域收入数据等分析区域的光污染水平\\
- 探讨光污染对该地区的多方面影响\\
- 使用GeoPandas和Folium等GIS可视化工具,展示了光污染的地理分布和严重程度,使研究结果更直观和易于理解\\
\textbf{SEO策略的博弈论分析:从方法到模型} \hfill \textit{2023年10月 - 2023年12月}\\
- 研究并实施多种搜索引擎优化(SEO)策略以提高网站排名\\
- 开发并验证了一种新的排名算法,结合了关键词频率、流量和链接\\
- 将博弈论原理应用于SEO,包括模拟$\alpha$-随机行走和纳什均衡分析\\
- 提出了一种多阶段策略以应对SEO的动态性\\
\textbf{基于AI的航班延误保险推荐系统} \hfill \textit{2024年2月 - 现在}\\
- 预测航班延误并推荐个性化旅行保险,以提高客户满意度\\
- 利用深度学习、自然语言处理和情感分析进行准确的延误预测和客户情感评估\\
- 通过客户反馈和情感分析,持续优化推荐系统,以提高客户满意度和保险购买率\\
\textbf{金融大型语言模型} \hfill \textit{2024年2月 - 现在}\\
- 根据新闻、公司公告等金融文本信息,使用大型语言模型进行情感分析,以预测股市趋势\\
- 设计了一个交互式代理,使基于情感的预测模型可供非技术用户使用\\
- 调优模型以优化预测准确性\\
\textbf{机器学习 \textit{(课程项目)}} \hfill \textit{2023年2月 -- 2023年5月}\\
- 基础理论学习,数据分析,数据可视化\\
- Python: numpy, pandas, matplotlib, sklearn, scipy 等\\
- 实现的模型:线性回归,支持向量机(SVM),决策树,K-均值聚类,主成分分析(PCA)等\\




\end{document}